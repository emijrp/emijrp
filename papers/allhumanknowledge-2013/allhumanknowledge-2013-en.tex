\documentclass[11pt,twocolumn]{article}
\setlength{\columnsep}{0.5cm}

\usepackage[utf8]{inputenc}
\usepackage[T1]{fontenc}
\usepackage[spanish]{babel}
%\usepackage{footmisc} % footers
\usepackage{hyperref} % to avoid colored links
% start hidelinks fixing
% as \usepackage[hidelinks]{hyperref}
% is broken
% \hypersetup is used instead
\hypersetup{
   colorlinks=false,
   pdfborder={0 0 0},
}
% end hidelinks fixing
\usepackage{graphicx}
\usepackage{natbib}

\title{\vspace{-15mm}
	\fontsize{24pt}{10pt}\selectfont
	\textbf{All Human Knowledge}
	}	
\author{
	\large
	\textsc{Emilio J. Rodríguez-Posada} \\
	\normalsize	Freelancer \\
	\normalsize	\href{mailto:emijrp@gmail.com}{emijrp@gmail.com}
	\vspace{-5mm}
	}
\date{}


%\pagestyle {myheadings}
%\markboth{Emilio J. Rodríguez-Posada}{WikiPapers: literatura sobre wikis recopilada colaborativamente... ¡en un wiki!}

\begin{document}

\twocolumn[
  \begin{@twocolumnfalse}

    \maketitle

\begin{abstract}
  The idea of compiling all human knowledge in a single work is highly seductive.
  \\
  \\
  \textbf{Keywords:} knowledge, encyclopedia, Wikipedia

\end{abstract}

  \end{@twocolumnfalse}
  ]

\section{Introduction}

\section{Related work}

Many individuals, groups and organizations have attempted to compile all human knowledge before (although many others focused only in a topic or set of topics). Some examples include (sorted by date): Library of Alexandria (3rd century BC) in Egypt, Naturalis Historia (AD 77–79) by Pliny the Elder, Speculum Maius (13th century) by Vincent of Beauvais, Bibliotheca universalis by Conrad Gessner (1545), the abstracting and indexing project by Gottfried Leibniz, L'Encyclopédie (1751–1772) by Diderot and d'Alembert, Mundaneum (1910) by Paul Otlet and Henri La Fontaine, Encyclopædia Britannica (1911) and more recently Interpedia (1993) by Rick Gates, Internet Archive (1996) by Brewster Kahle and Wikidata (2012). Hypothetical cases exist: Encyclopædia Galactica (1980) by Carl Sagan in Cosmos, Permanent World Encyclopaedia (1936–1938) by H. G. Wells and Memex (1945) by Vannevar Bush. Finally, there are imaginary examples too: "The Universal Library" (1901) by Kurd Lasswitz, "The Total Library" essay and The Library of Babel (1941) by Jorge Luis Borges, Encyclopædia Galactica (1942) by Isaac Asimov in "Foundation" and the Akashic records.

\section{All Human knowledge}

\section{Destroyed knowledge}

\section{Conclusions and related work}

%\bibliographystyle{wink}
\bibliographystyle{hunsrtnat}
\bibliography{allhumanknowledge-2013}

\section*{Acknowledgements}



\section*{License}
This work is licensed under a \href{http://creativecommons.org/licenses/by-sa/3.0/}{Creative Commons Attribution-ShareAlike 3.0 Unported}.

\end{document}
