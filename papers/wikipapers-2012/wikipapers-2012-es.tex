\documentclass[11pt,twocolumn]{article}
\setlength{\columnsep}{0.5cm}

\usepackage[utf8]{inputenc}
\usepackage[T1]{fontenc}
\usepackage[spanish]{babel}
\usepackage{hyperref}
\usepackage{graphicx}
\usepackage{natbib}

\title{\vspace{-15mm}%
	\fontsize{24pt}{10pt}\selectfont
	%\textbf{WikiPapers: a collaborative compilation of wiki research literature... in a wiki!}
	\textbf{WikiPapers: publicaciones sobre wikis recopiladas colaborativamente... ¡en un wiki!}
	}	
\author{%
	\large
	\textsc{Emilio J. Rodríguez-Posada} \\
	\normalsize	Private \\
	\normalsize	\href{mailto:emijrp@gmail.com}{emijrp@gmail.com}
	\vspace{-5mm}
	}
\date{}


\begin{document}


\twocolumn[
  \begin{@twocolumnfalse}

    \maketitle

\begin{abstract}
  El interés de los investigadores por los wikis, en especial Wikipedia, ha ido creciendo en los últimos años. La primera edición de WikiSym, un simposio sobre wikis, se celebró en 2005 y desde entonces han aparecido congresos, workshops, conferencias y competiciones en este área. El estudio de los wikis es un campo emergente y prolífico. Ha habido varios intentos, aunque con escaso éxito, de recopilar toda la literatura sobre wikis. Unas veces el enfoque o la herramienta utilizada eran limitados, otras el proyecto era abandonado y al poco tiempo los metadatos bibliográficos se perdían. En este artículo presentamos WikiPapers, un proyecto colaborativo para recopilar toda la literatura sobre wikis. Se hace uso de MediaWiki y su extensión semántica, ambos conocidos por los investigadores de este campo. Hasta octubre de 2012 se han recopilado más de 1.400 publicaciones y sus metadatos, además de documentación sobre herramientas y datasets relacionados. Los metadatos son exportables en los formatos BibTeX, RDF, CSV y JSON. Los historiales completos del wiki están disponibles para descarga.
\end{abstract}

  \end{@twocolumnfalse}
  ]
	

\section{Introducción}
El interés de los investigadores por los wikis, en especial Wikipedia, ha ido creciendo en los últimos años. La primera edición de WikiSym, un simposio sobre wikis, se celebró en 2005 y desde entonces han aparecido congresos como CLEF/PAN Lab, workshops como WikiAI, SemWiki y MathWikis, conferencias como Wikimania, WikiCon, SMWCon, Wiki Conference India, Wikipedia Academy y Wikipedia CPOV Conference y competiciones como WikiViz. El estudio de los wikis es un campo emergente y prolífico.

El resto del artículo se divide de la siguiente manera. En la sección 2 hacemos un repaso a los distintos enfoques usados para recopilar toda la literatura sobre wikis, sus ventajas e inconvenientes. En la sección 3 presentamos WikiPapers, cómo funciona y qué pasos se han dado. Finalmente terminamos con unas conclusiones y trabajo futuro en la sección 4.

\section{Trabajos relacionados}
Ha habido varios intentos de recopilar toda la literatura sobre wikis, aunque con poco éxito. Unas veces el enfoque o la herramienta utilizada eran limitados, otras veces el proyecto era abandonado y al poco tiempo los metadatos bibliográficos se perdían. Se han hecho recopilaciones en páginas personajes y blogs, a través de revisiones de literatura, haciendo uso de gestores de bibliografía, en páginas de Wikipedia y también en servicios como Zotero o CiteULike. Evaluamos sus ventajas e inconvenientes y cómo WikiPapers resuelve estos problemas.

\subsection{Páginas personales y blogs}
Existen ejemplos de recopilaciones de literatura en webs personales\footnote{\href{http://www.public.iastate.edu/~CYBERSTACKS/WikiBib.htm}{http://www.public.iastate.edu/~CYBERSTACKS/WikiBib.htm}} y blogs. Un ejemplo bastante completo de este último es SWEETpedia,\footnote{\href{http://www.mkbergman.com/sweetpedia/}{http://www.mkbergman.com/sweetpedia/}} que contiene publicaciones sobre wikis y semántica. Uno de los inconvenientes de este sistema es que el esfuerzo suele recaer sobre una única persona y los metadatos no son fácilmente exportables. En WikiPapers la revisión se hace colaborativamente y todos los metadatos son fácilmente exportables en diversos formatos.

\subsection{Revisiones de literatura}
Se han realizado varias revisiones de literatura hasta el momento. La primera de ellas ~\citep{voss2005} se hizo en un momento en el que las publicaciones eran escasas, pero ya se intuía que estaba en crecimiento. Un año más tarde (Ayers, 2006) vuelve a hacer un repaso a la literatura existente.

No sería hasta 3 años después cuando ~\citep{okoli2009b} presentan una propuesta de protocolo para hacer un mapeo sistemático y ese mismo año ~\citep{okoli2009} analiza el estdo del arte.

~\citep{nielsen2011} hace la mayor revisión de literatura en un documento de más de 50 páginas, en progreso e inacabado, que incluye 300 referencias a publicaciones y asegura haber encontrado más de 1.000 publicaciones sobre el tema.

~\citep{martin2011}

~\citep{okoli2012}

~\citep{jullien2012}

Uno de los inconvenientes de estas revisiones de literatura es que quedan rápidamente desactualizadas debido al ritmo con el que aparecen nuevas publicaciones. WikiPapers está abierto a ampliaciones y mejoras y es actualizado continuamente.

\subsection{Gestores bibliográficos}
Se han empleado gestores bibliográficos específicos como WIKINDX\footnote{\href{http://sourceforge.net/projects/wikindx/}{http://sourceforge.net/projects/wikindx/}} creando portales como Wikibibliographie ENCYCLEN\footnote{\href{http://wikindx.inrp.fr/biblio_encyclen/}{http://wikindx.inrp.fr/biblio\_encyclen/}} pero decidieron restringir la edición a un círculo de usuarios aprobados. En WikiPapers pueden participar tanto usuarios registrados como sin registrar.

%\href{http://toolserver.org/~voj/bibliography/}{http://toolserver.org/~voj/bibliography/}
%\href{http://wikiindex.org/Wiki_Research_Bibliography}{http://wikiindex.org/Wiki\_Research\_Bibliography}

\subsection{Páginas individuales en Wikipedia}
También existen listados de publicaciones y recursos en algunas Wikipedias, como en la versión alemana\footnote{\href{http://de.wikipedia.org/wiki/Wikipedia:Wikipedistik/Bibliographie}{http://de.wikipedia.org/wiki/Wikipedia:Wikipedistik\\ /Bibliographie}} y la inglesa\footnote{\href{http://en.wikipedia.org/wiki/Wikipedia:Academic_studies_of_Wikipedia}{http://en.wikipedia.org/wiki/Wikipedia:Academic\\ \_studies\_of\_Wikipedia}}. El principal inconveniente es que no es posible jugar con los datos dentro del mismo wiki, al estar todo escrito como texto plano, sin enriquecimiento semántico.

\subsection{Servicios web y redes sociales}
Finalmente servicios web y redes sociales con recopilaciones de literatura sobre wikis. Es el caso de grupos de Zotero,\footnote{\href{https://www.zotero.org/groups/wikipedia_research}{https://www.zotero.org/groups/wikipedia\_research}} tags de BibSonomy\footnote{\href{http://www.bibsonomy.org/tag/wikipedia}{http://www.bibsonomy.org/tag/wikipedia} y \href{http://www.bibsonomy.org/tag/wiki}{http://www.bibsonomy.org/tag/wiki}} y grupos y tags de CiteULike.\footnote{\href{http://www.citeulike.org/tag/wikipedia}{http://www.citeulike.org/tag/wikipedia}, \href{http://www.citeulike.org/tag/wiki}{http://www.citeulike.org/tag/wiki} y \href{http://www.citeulike.org/group/382}{http://www.citeulike.org/group/382}}

\section{WikiPapers}
WikiPapers\footnote{\href{http://wikipapers.referata.com}{http://wikipapers.referata.com}} fue lanzado en abril de 2011. Haciendo uso de MediaWiki y su extensión semántica, recopila de manera colaborativa información acerca de toda la literatura científica sobre wikis, así como de herramientas y datasets relacionados. No hace falta estar registrado para participar, pero es recomendable.

WikiPapers agrupa todas las ventajas de los sistemas mencionados anteriormente y soluciona sus inconvenientes. Permite hacer listados de publicaciones similares a SWEETpedia: existe uno de revisiones de literatura por poner solo un ejemplo.\footnote{\href{http://wikipapers.referata.com/wiki/List_of_literature_reviews}{http://wikipapers.referata.com/wiki/List\_of\\ \_literature\_reviews}} Funciona como un gestor bibliográfico, al almacenar los metadatos de las publicaciones y permitir hacer búsquedas, filtrarlos o exportarlos, individualmente o en conjunto. También facilita que grupos de usuarios se comuniquen a través de las páginas de discusión y compartan información sobre publicaciones de su interés, funcionando como una red social. El espacio de discusión debajo de cada página posibilita a los lectores hacer valoraciones de los artículos.

Desde un punto de vista más estadístico, es posible generar gráficas a partir de los metadatos disponibles en WikiPapers, aprovechando así la capacidad que ofrece la semántica. Gráficos de barras, circulares o líneas temporales están presentes y facilitan la visualización y comprensión de la información. Existe la posibilidad de incrustar diapositivas (SlideShare) y vídeos (YouTube, Vimeo).

Finalmente, el wiki y sus historiales están disponibles tanto para su descarga como dump XML y accesible a través de la API de MediaWiki. Esto impide que todo el trabajo se pierda, como ha sucedido en otros proyectos.

\begin{figure}[htb]
\centering
\includegraphics[width=0.49\textwidth]{wpfull.png}
\caption{Portada de WikiPapers}
\label{fig:wpfull}
\end{figure}

\subsection{Publicaciones}
En WikiPapers cada publicación dispone de una página en la que se detallan todos sus metadatos (título, autores, palabras clave, año, revista o congreso, DOI, idioma, licencia, enlaces al fichero y motores de búsqueda), el abstract, las referencias que incluye y las citas que recibe, y un espacio de discusión. Los metadatos sirven para hacer búsquedas y filtrar los contenidos. A octubre de 2012 ya cuenta con más de 1.400 publicaciones, incluyendo artículos de revistas y congresos, tesis y libros.

\subsection{Autores}
Para cada autor existe una ficha que incluye su nombre, afiliación, país, índice de coautores, página web, estadísticas sobre número de publicaciones y citas, y por supuesto un listado de publicaciones, datasets y herramientas de su creación.

\subsection{Herramientas}
Se está construyendo un listado de herramientas\footnote{\href{http://wikipapers.referata.com/wiki/List_of_tools}{http://wikipapers.referata.com/wiki/List\_of\_tools}} que se han desarrollado y usado sobre wikis, entre las que se incluyen:

\begin{itemize}
\item \textbf{Anti-vandalismo:} AVBOT, Clue Bot, CryptoDerk's Vandal Fighter, Huggle, Igloo, STiki, Salebot, Twinkle, Vandal Fighter, VandalProof, VandalSniper.
\item \textbf{Estadística y visualización:} HistoryFlow, StatMediaWiki, Wiki Explorator, Wiki Trip, WikiEvidens, WikiTrust.
\item \textbf{Framework:} Java Wikipedia Library, Perlwikipedia, pywikipedia.
\item \textbf{Lenguaje:} Manypedia, Wikokit, Zawilinski.
\item \textbf{Preservación:} WikiTeam tools.
\item \textbf{Procesamiento de datos:} DiffDB, Ikiwiki, Infobox2rdf, Wiki Edit History Analyzer, Wikihadoop, Wikipedia Miner.
\end{itemize}

La lista no es exhaustiva y está en continuo crecimiento.

\subsection{Datasets}
Repaso a los datasets, los wikis como datasets (WikiTeam)

\subsection{Y más...}
Además de todo lo anterior, también se está recopilando información sobre conceptos, ejemplos, preguntas abiertas, encuestas, motores wiki, wikifarms y más. WikiPapers se presenta como un lugar donde investigadores puedan hacer comunidad y establecer conexiones para futuras investigaciones. Los autores pueden utilizar sus páginas de usuario para presentarse.

\subsection{Reutilización}
Posibilidades de reutilizar el contenido ….
Todos los metadatos se pueden exportar en los formatos BibTeX, RDF, CSV y JSON.
Comentar que WikiLit ha cogido las plantillas y estructura de WikiPapers para crear su propio review de la literatura, limitado a revistas... ?

\section{Conclusiones y trabajo futuro}
porqué hacía falta WikiPapers
aglutina todas las ventajas de los anteriores sistemas
lo que se ha hecho, cifras,
lo que queda por hacer y como ayudar
el futuro y más allá...

\bibliographystyle{wink}        
\bibliography{wikipapers-2012}

\section{Licencia}
Esta obra está bajo licencia \href{http://creativecommons.org/licenses/by-sa/3.0/}{Creative Commons Reconocimiento-CompartirIgual 3.0 Unported}.

\end{document}
